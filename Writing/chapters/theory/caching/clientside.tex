\subsection[Client Side Caching]{Client Side Caching}
People find it annoying when it takes too long or laggy to load a website they want to access. 
This is where client side caching could be benificial.
Client side caching is a technique where data is cached inside the client's device unlike the application level caching. 
Frequently used static data are normally cahced on the client side to improve the performance and efficiency of 
web applications. The redueced time to load the webpages result in enhanced user's s tisfication and the overall experience.
% https://www.geeksforgeeks.org/system-design/server-side-caching-and-client-side-caching/

Cache control

One type of client side caching is browser cache. The browser can cache static files such as HTML, CSS, Javascript, 
and even images and fonts. The browser usually automatically decides what to cache, but this is dependant on by the 
HTTP cache-control header. 
Here are the 15 types of the headers that are accepted:
% https://www.geeksforgeeks.org/computer-networks/http-headers-cache-control/
% https://developer.mozilla.org/en-US/docs/Web/HTTP/Reference/Headers/Cache-Control

Etags

