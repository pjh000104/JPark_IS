\section[Client Side Caching]{Client Side Caching}
People find it annoying when it takes too long or laggy to load a website they want to access.
This is where client side caching could be beneficial.
Client side caching is a technique where data is cached inside the client's device unlike the application level caching.
Frequently used static data are normally cached on the client side to improve the performance and efficiency of
web applications. The reduced time to load the webpages result in enhanced user's satisfaction and the overall experience.
Static data could also be available when the user's device is offline, which could be useful for some mobile and web applications. 
% https://www.geeksforgeeks.org/system-design/server-side-caching-and-client-side-caching/

% Types of Client Side Caching:
% browser, service workers, local storage 




There are different types of client side caching that could be used, but the most common ones are local storage, session storage, and browser cache.
Local storage and session storage are types of web storage APIs, where developers can store data on the client side that is not automatically cached by the browser.
This could include saving users' preferences such as theme or language, or simple offline data. The difference between local storage and 
Session storage is the scope of the browser. Local storage is shared among all tabs and windows of the same origin, but session storage is specific to a single tab.
For session storage, they will be deleted once a tab is closed. However, local storage will be saved even after the tab and the browser are closed.

Browser cache is where most of the static elements of the web application, such as HTML, CSS, JavaScript, and images, are automatically cached.
The way how and where these caches are stored depend on the browser the user is using, but the core concepts are the same.
Although browser cache is automatically stored, there are some parts that could be controlled within the browser cache, which is called cache control.
Cache control is an HTTP response header that gives instructions to the browser on how to cache resources. 
There are 15 total cache control techniques, and each of these techniques serves different purposes,
depending on how often the data changes and how much freshness the application requires, but we will be discussing the most common ones. 


% The top five strategies that would be most used in myy application are the following: 
% cache forever with immutable, no-cache, no-store, short-lived caching, and stale-while-revalidate.

% The most common strategy is cache forever with immutable. This is mostly used for static files like JavaScript,
% CSS, images, and fonts. Many projects use hashed filenames, so the file name changes when the content does.
% This lets developers allow browsers to store these files for a long time without worrying about outdated content.
% The immutable option tells the browser the file will not change, so it can skip extra checks and load pages faster for returning users.

   
% One type of client side caching is browser cache. The browser can cache static files such as HTML, CSS, Javascript, 
% and even images and fonts. The browser usually automatically decides what to cache, but this is dependant on by the 
% HTTP cache-control header. 
% Here are the 15 types of the headers that are accepted:
% https://www.geeksforgeeks.org/computer-networks/http-headers-cache-control/
% https://developer.mozilla.org/en-US/docs/Web/HTTP/Reference/Headers/Cache-Control


