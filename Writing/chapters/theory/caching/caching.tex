% \section[Caching]{Caching}\label{sec:newsec}
\chapter{Caching}\label{text}

\section[What is Caching]{What is Caching}
Caching data means to store frequently accessed data in a location that is easy and fast to access (usually in RAM). 
The reason why caching is used is to ultimately increase the performance and efficiency when retrieving data. 
Data in the database are usually stored in a disk, which could be a hard drive or an SSD, which usually takes more time to process I/O. 
Caching could help decrease the amount of time to process this data significantly. One might suggest then why not put all the data in the cache, 
but this would not work since the cache could not store a lot of data as much as the database does. The cost of the hardware of the cache is much more expensive, 
and as the size of the cache increases, the search time will also increase, decreasing the speed and defeating the whole purpose of caching. 
Caching can be used at several levels. This article will mainly focus on caching in databases and web pages. 
There are multiple strategies that could be used when caching these two areas. Another important part of caching is the invalidation strategies. 
Deciding when the cache expires and when it updates could affect the performance and data consistency. Choosing the right technique in the given situation will be essential.

\section[Database Caching Strategies]{Database Caching Strategies}

There are multiple database caching strategies that could be applied when designing cache systems.
Depending on the type of request (read, write) of the user, and which part of the application is responsible for fetching data from the database or managing the cache,
there are five main database caching strategies.\\


% \subsection{Cache Aside}
\noindent \textbf{Cache Aside}

Cache Aside is also called Lazy Loading. The application is in control of managing the cache. Let’s look at the diagram below.

% \begin{center}
% (put cache aside diagram)
% \begin{figure}[H]
    \centering
    \includegraphics[width=0.8\textwidth]{cache-aside.png} % your file name here
    \caption{Entity-Relationship Diagram}
    \label{fig:erdiagram}
\end{figure}
% \end{center}

\providecommand{\cacheAsideFigure}{
\begin{figure}[H]
    \centering
    \includegraphics[width=0.8\textwidth]{myFigures/cache-aside.png}
    \caption{Model of Cache-aside Pattern to explain common database caching strategies \cite{Thiraviyam_2025}.}
    \label{fig:cache-aside}
\end{figure}
}

\providecommand{\readThroughFigure}{
\begin{figure}[H]
    \centering
    \includegraphics[width=0.8\textwidth]{myFigures/read-through.png}
    \caption{Model of Read-Through Pattern to explain common database caching strategies \cite{Thiraviyam_2025}.}
    \label{fig:read-through}
\end{figure}
}

\providecommand{\writeAroundFigure}{
\begin{figure}[H]
    \centering
    \includegraphics[width=0.8\textwidth]{myFigures/write-around.png}
    \caption{Model of Write-Around Pattern to explain common database caching strategies \cite{Thiraviyam_2025}.}
    \label{fig:write-around}
\end{figure}
}

\providecommand{\writeBackFigure}{
\begin{figure}[H]
    \centering
    \includegraphics[width=0.8\textwidth]{myFigures/write-back.png}
    \caption{Model of Write-Back Pattern to explain common database caching strategies \cite{Thiraviyam_2025}.}
    \label{fig:write-back}
\end{figure}
}

\providecommand{\writeThroughFigure}{
\begin{figure}[H]
    \centering
    \includegraphics[width=0.8\textwidth]{myFigures/write-through.png}
    \caption{Model of Write-Through Pattern to explain common database caching strategies \cite{Thiraviyam_2025}.}
    \label{fig:write-through}
\end{figure}
}

\providecommand{\binaryTreeFigure}{
\begin{figure}[H]
    \centering
    \includegraphics[height=5cm]{myFigures/binaryTree.png}
    \caption{Model of a binary tree to explain datastructures used for database indexing.}
    \label{fig:binaryTree}
\end{figure}
}

\providecommand{\bTreeFigure}{
\begin{figure}[H]
    \centering
    \includegraphics[height=5cm]{myFigures/bTree.png}
    \caption{Model of a b tree to explain datastructures used for database indexing.}
    \label{fig:bTree}
\end{figure}
}

\providecommand{\bPlusTreeFigure}{
\begin{figure}[H]
    \centering
    \includegraphics[height=5cm]{myFigures/b+Tree.png}
    \caption{Model of a b+ tree to explain datastructures used for database indexing.}
    \label{fig:b+Tree}
\end{figure}
}

\providecommand{\leaderFollowerFigure}{
\begin{figure}[H]
    \centering
    \includegraphics[width=0.8\textwidth]{myFigures/leaderFollower.png}
    \caption{Figure of Leader and Follower replication strategy to explain the flow or write or read database querie \cite{Kleppmann_2017}.}
    \label{fig:leader follower}
\end{figure}
}

\providecommand{\erdFigure}{
\begin{figure}[p]  % 'p' = float on a page by itself
    \centering
    \includegraphics[width=\paperwidth,height=\paperheight,keepaspectratio]{myFigures/erd.png}
    \caption{Figure of the entity relationship diagram for the local cuisine application.}
    \label{fig:erd}
\end{figure}
}

\cacheAsideFigure

As the diagram shown above, the client first checks the cache to see if the read request data is in the cache.
If the cache exists, this is called a cache hit, and the client uses it right away.
If the cache does not exist, this is called a cache miss. When a cache miss occurs, the client then sends a request to the database server to fetch the data.
After that, the client transfers the data to the cache, so that there could be a cache hit the next time the data is needed.
This will increase the performance of the application when the same data is called multiple times.
When there is a write request, the application will first communicate with the database and then update the cache.

This strategy is useful when the application requires a lot of read requests from the database,
since the application could just check the cache instead of querying the database.
One disadvantage of this approach is that there could be an inconsistency between the cache and the database.
Data directly written in the database might not be consistent with the data in the cache,
since writing data to the database and writing data to the cache does not happen at the same time.\\

% \subsection{Read-Through}
\noindent \textbf{Read-Through}

In the read-through strategy, the cache level manages fetching data from the database. Here is a diagram of how the read-through strategy works:

% \begin{center}
% (put read-through diagram)
% \end{center}

\readThroughFigure

Whenever there is a read request from the application, the application first checks the cache. If there is a cache hit, it simply returns the data back to the application. If there is a cache miss, the cache level fetches the data from the database. Since the cache manages fetching data, it simplifies the application logic when retrieving data compared to when the application handles fetching data.

This strategy only involves read requests from the application, which means that other strategies could be used for write requests.
The write-through strategy is generally great to be used with the read-through strategy.\\

% \subsection{Write-Through}
\noindent \textbf{Write-Through}

The write-through strategy is very similar to the read-through strategy. The diagram is almost identical.

% \begin{center}
% (put diagram)
% \end{center}
\writeThroughFigure

Nothing happens when there is a read request from the application, but when there is a write request, the data will first be written in the cache and then into the database. This ensures data consistency when paired with the read-through strategy. One downside of this strategy is that since it writes to both the cache and the database layer, the latency of the request increases.

The read-through and the write-through strategies are not meant to be used for all access to data in the application layer.
Using these strategies for all database queries could result in a decrease in performance, since the caching layer is not intended to keep every single data.\\

% \subsection{Write-Back}
\noindent \textbf{Write-Back}

The write-back strategy is similar to the write-through strategy, but the writes to the database happen with a delay. The cache is still in charge of writing data to the database. Let’s look at the diagram:

% \begin{center}
% (write-back diagram)
% \end{center}
\writeBackFigure

Since the cache only writes to the database after a certain amount of time,
this strategy is beneficial when there are multiple write requests to the same data at the application level.
This will decrease the write queries from the cache level to the database, which could increase the performance.
However, similar to the read-through and write-through strategies, since the data is written to the cache first,
writing data to the database could potentially fail if the caching fails.\\

% \subsubsection{Write Around}
\noindent \textbf{Write Around}

This strategy is similar to the cache-aside strategy, but it has more specific instructions for write requests.

% \begin{center}
% (write-around diagram)
% \end{center}
\writeAroundFigure

The difference between the write-around strategy and the cache-aside strategy is that when there is a write request, only the data in the database is updated. The data in the cache is only updated when there is a cache miss from a read request. This way, the cache is not overflooded with unnecessary data. The disadvantage of this strategy is that recently written data will always result in a cache miss. This is why the write-around strategy is suitable when the application has a lot of read requests and data is rarely updated.

\section[Redis]{Redis}
% https://learning.oreilly.com/library/view/redis-in-action/9781617290855/kindle_split_011.html
Until now, we have learned the types of database caching strategies. But how are these strategies actually implemented?
Redis is a widely used open-source data structure store that makes these caching patterns practical at scale.

Redis works entirely in memory, which gives it extremely fast read and write speeds.
Most databases store data on disk, so every query involves slower I/O operations.
Redis avoids this by keeping data in RAM, allowing applications to access cached values in microseconds.
This speed makes it perfect for handling database caching strategies \cite{Carlson_2013}.

Redis also supports a wide range of data structures—strings, hashes, lists, sets, sorted sets, and more.
These structures map naturally to different caching scenarios. For example, strings work well for caching simple query results,
hashes can store user profile objects, and sorted sets can maintain real-time rankings or leaderboards.
Because Redis supports many data structures, it is not limited to simple key–value caching.
It can also function as a fast and lightweight data engine that handles situations that require complex caching.

Reliability is another reason Redis is chosen for database caching. Redis provides persistence options (RDB snapshots and AOF logs)
so cached data or application-critical information can survive server restarts. It also supports replication and clustering,
which distribute data across multiple nodes to achieve both high availability and horizontal scalability.
These features ensure that the caching layer does not become a single point of failure.

In summary, Redis transforms abstract caching strategies into practical and high-performance solutions.
Its low latency, adaptable data model, and robust operational features make it a dependable component in web applications.
These are the reasons why Redis will be used to implement database caching strategies in this project.

\subsection[Client Side Caching]{Client Side Caching}
Redis

\section[Caching Invalidation]{Caching Invalidation}
After a cache is stored whether in the client or the server side, it is crucial to know when the cache should be validated.
For example, let's assume the user's profile is cached in the browser. Without any caching invalidation, even after the user's
profile is updated, the user will not be able to see the changes if the cache still exists in the browser. 
This could lead to crucial data inconsistency as the importance of the data increases such as sensitive security data.
Caching invalidation strategies are needed to set when the cache should be stored, updated, or deleted.
Choosing the right caching invalidation strategy can lead to better performance of the application, freeing up the memory space
and increasing cache-hit rates. 

% I will be explaining about dependency ID first, since it is a term that will be mentioned often in the following paragraphs.
% A dependency ID is a label used to identify which cache should be invalidated, and the same label can be attached to multiple cache entries.
% By doing this, group is created that can be removed all together when certain conditions are met. The conditions will be the different types 
% of caching validation which will be explained in the pargarphs bellow. This group allows the developers to structure cache making
% caching invalidation easier and more organized. 

One of the most common types of cache invalidation is time-based invalidation. This type of technique invalidates cached data
after a certain time period has passed, which could be decided by the developers \cite{Redis_2025}.
Although it is simple and widely used, it also has limitations because it may invalidate data too early or too late depending on the situation.
In some systems, a timeout value can be added directly in a cache configuration file to specify how long the entry should stay valid.
This value can also be customized to fit the needs of the application. 
Some examples of time-based invalidation could be a recipe website may refresh its cache daily to show the latest trending dishes, 
or a movie streaming service that could be refreshed where there are new releases every few hours.

Another type of invalidation is event-based invalidation, which happens when a specific event is triggered in the system.
This is useful when the cached data is tied closely to an event or state change.
For example, when a blog post is updated, the old cached version of the post should be removed so that users can see the most recent version right away.
This ensures that the content stays consistent with changes happening inside the system.

Another type is group-based invalidation, which removes cache based on a larger group or category.
This is helpful when many cache entries belong to the same section of the application, and clearing them one by one would be inefficient \cite{Redis_2025}.
This type of cache invalidation technique is commonly used when a larger group of data must be invalidated at once.
For example, on a news website, if the politics section is updated, all cached articles under that section should be
invalidated to make sure the latest content is displayed. In eCommerce systems, updating a product category may require invalidating all caches related to that category.

Lastly, there is validation-based invalidation. This cache invalidation technique relies on the browser checking with the server to see if
the cached data is still valid instead of removing it after a fixed time or when an event occurs \cite{Redis_2025}.
A common technique for this is using ETags, which are small identifiers generated by the server to represent a specific version of a resource.
When the browser stores a cached file, it also keeps the ETag value and later sends it back to the server to confirm whether the resource has changed.
If it has not changed, the server replies with a “Not Modified” response, which allows the browser to continue using the cached version without downloading it again.
This approach helps reduce unnecessary network usage while still ensuring that users receive the most up-to-date content,
making it useful for situations where data changes unpredictably.

Not all of these caching strategies could be used both in the server side and the client side. 
Each technique serves a different purpose, and some are only effective when controlled by the system that manages the data.
Time-based invalidation is mostly handled on the server side through cache headers,
while event-based and group-based invalidation occur exclusively on the server since they depend on changes happening within the application itself \cite{Redis_2025}.
On the other hand, validation-based methods such as ETags rely on collaboration between the browser and the server,
where the server generates validation tokens and the client uses them to check freshness.
Understanding where each technique is applied helps ensure that caching remains efficient, consistent, and aligned with the needs of the application.

% Another common technique is no-cache, which still allows the browser to cache the resource,
% but requires it to check with the server before using the cached version.
% This is useful for files that change more frequently, such as HTML documents or some API responses.
% Even though the browser revalidates the file, it does not have to download the whole resource again unless it has changed.
% This helps balance performance while still keeping the content up to date.

% The no-store is a caching invalidation technique when there is no need for the browser to store data in any type of cache.
% Because of this the no-store technique is commonly used for sensitive or private data.  
% This is important for pages that include personal information, financial data, or authentication flows.
% By preventing the browser from saving these resources, the application can reduce the risk of exposing sensitive data on the client side.

% Short-lived caching is another practical approach where resources are cached for a short amount of time.
% The key difference between this technique and the time-based cache invalidation technique is the amount of time the cache is stored.
% Short-lived caching stores cache a few seconds or minutes, while time-based caching is generally more longer. 
% This strategy works well for information that does not need to be real-time but still updates often,
% like live sports games, news feeds, or dashboard statistics. Using short-lived caching can reduce the number of requests sent to the server,
% while still keeping the user experience smooth. 


% https://redis.io/glossary/cache-invalidation/