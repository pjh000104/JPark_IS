\section[Optimization]{Optimization}
\providecommand{\cacheAsideFigure}{
\begin{figure}[H]
    \centering
    \includegraphics[width=0.8\textwidth]{myFigures/cache-aside.png}
    \caption{Model of Cache-aside Pattern to explain common database caching strategies \cite{Thiraviyam_2025}.}
    \label{fig:cache-aside}
\end{figure}
}

\providecommand{\readThroughFigure}{
\begin{figure}[H]
    \centering
    \includegraphics[width=0.8\textwidth]{myFigures/read-through.png}
    \caption{Model of Read-Through Pattern to explain common database caching strategies \cite{Thiraviyam_2025}.}
    \label{fig:read-through}
\end{figure}
}

\providecommand{\writeAroundFigure}{
\begin{figure}[H]
    \centering
    \includegraphics[width=0.8\textwidth]{myFigures/write-around.png}
    \caption{Model of Write-Around Pattern to explain common database caching strategies \cite{Thiraviyam_2025}.}
    \label{fig:write-around}
\end{figure}
}

\providecommand{\writeBackFigure}{
\begin{figure}[H]
    \centering
    \includegraphics[width=0.8\textwidth]{myFigures/write-back.png}
    \caption{Model of Write-Back Pattern to explain common database caching strategies \cite{Thiraviyam_2025}.}
    \label{fig:write-back}
\end{figure}
}

\providecommand{\writeThroughFigure}{
\begin{figure}[H]
    \centering
    \includegraphics[width=0.8\textwidth]{myFigures/write-through.png}
    \caption{Model of Write-Through Pattern to explain common database caching strategies \cite{Thiraviyam_2025}.}
    \label{fig:write-through}
\end{figure}
}

\providecommand{\binaryTreeFigure}{
\begin{figure}[H]
    \centering
    \includegraphics[height=5cm]{myFigures/binaryTree.png}
    \caption{Model of a binary tree to explain datastructures used for database indexing.}
    \label{fig:binaryTree}
\end{figure}
}

\providecommand{\bTreeFigure}{
\begin{figure}[H]
    \centering
    \includegraphics[height=5cm]{myFigures/bTree.png}
    \caption{Model of a b tree to explain datastructures used for database indexing.}
    \label{fig:bTree}
\end{figure}
}

\providecommand{\bPlusTreeFigure}{
\begin{figure}[H]
    \centering
    \includegraphics[height=5cm]{myFigures/b+Tree.png}
    \caption{Model of a b+ tree to explain datastructures used for database indexing.}
    \label{fig:b+Tree}
\end{figure}
}

\providecommand{\leaderFollowerFigure}{
\begin{figure}[H]
    \centering
    \includegraphics[width=0.8\textwidth]{myFigures/leaderFollower.png}
    \caption{Figure of Leader and Follower replication strategy to explain the flow or write or read database querie \cite{Kleppmann_2017}.}
    \label{fig:leader follower}
\end{figure}
}

\providecommand{\erdFigure}{
\begin{figure}[p]  % 'p' = float on a page by itself
    \centering
    \includegraphics[width=\paperwidth,height=\paperheight,keepaspectratio]{myFigures/erd.png}
    \caption{Figure of the entity relationship diagram for the local cuisine application.}
    \label{fig:erd}
\end{figure}
}

Several factors influence the performance of the database.
The design of the database, including normalization and partitioning, which was discussed in the previous subsection, is one of the factors.
Indexing is another huge part of increasing the performance of the database as the amount of data increases.
Lastly, changing the physical query of the database is a way to optimize the performance of the database.
In this section, we will be discussing specific ways to optimize the performance of databases.

\subsection{Indexing}

Suppose that there is a table called students with the following keys: id, name, grade.
Here is an example table: 
\begin{table}[H]
\centering
\begin{tabular}{|c|c|c|}
\hline
\textbf{ID(PK)} & \textbf{Name} & \textbf{Grade} \\ \hline
1 & Alice   & 90 \\ \hline
2 & Bob     & 80 \\ \hline
3 & Charlie & 70 \\ \hline
4 & David   & 70 \\ \hline
5 & Eva     & 80 \\ \hline
\end{tabular}
\caption{Original Students Table with ID, Name, and Grade}
\label{tab:students_original}
\end{table}

If we want to find students with a certain grade (80\%),
we could run a SQL query something like this.
\begin{verbatim}
SELECT * 
FROM Students
WHERE grade = 80;
\end{verbatim}

This query will visit all the rows and then find the students with grades 80. However, this query becomes more inefficient as the size of the table grows,
as the time complexity would be O(n). If the table was sorted by grades, this will be much easier, since we could do a binary search to make the search speed O(log n).
This is where indexing is applied. 

An index could be created where the grades are sorted in order, each having a reference to the row of the student table.
This way, whenever we want to find the name of the students with a certain grade, instead of looking at the Students table, we can look at the index,
where the grades are sorted, find rows with a certain grade, for each row, find the reference of the student table, and return the name of the student.
Here is an example how the index might look like:

\begin{table}[H]
\centering
\begin{tabular}{|c|c|c|}
\hline
\textbf{Grade} & \textbf{ID}  \\ \hline
70 & 3  \\ \hline
70 & 4  \\ \hline
80 & 2  \\ \hline
80 & 5  \\ \hline
90 & 1  \\ \hline
\end{tabular}
\caption{Index on Grade: sorted by Grade with references to Original Table}
\label{tab:students_index}
\end{table}

In this index, grades are sorted by order, and they include a reference to the student table's primary key, which is the student's id.
This will significantly improve the read queries for databases for large sets of data.

However, indexes take up additional memory and can slow down write operations, such as inserts, updates, or deletes. It is important to use indexes on every column.
Indexes should be used in situations where they could significantly increase the performance of read queries.

There are multiple data structures that could be used to implement indexes,
but the most common ones are B+trees. In order to understand why B+trees are used the most we have to learn what are B trees.
B trees are somewhat similar to the binary search trees.
However, instead of having one value for every node, it contains multiple values for each node.
\binaryTreeFigure
\bTreeFigure

As we can see from the figures above, B trees can hold multiple nodes, such as nodes 2 and 4, or 10 and 15.
This allows multiple partitions, allowing a balanced and faster search.
Furthermore, as multiple values could be in a single node, the height of the tree would be smaller.
This means that there are fewer disk I/O’s per file operation as the database stores data on disk, resulting in faster query execution \cite{Garcia-Molina_Ullman_Widom_2014}.
% (pg 645 Database Systems the Complete Book). 
One down side of B trees is when making range queries, such as finding all students who have grades 20 to 80.
If these two data are separated from the root, this could be inefficient. B+trees are a variation of B trees to solve this issue.
\bPlusTreeFigure
B+trees store data only in the leaf node, which are the bottom most nodes without any children.
The other nodes store keys only for navigation, like the explanations on the internal node 
(less than equal to six goes left, greater than six goes right) on the figure above. Also, all the leaf nodes are connected to the next leaf node.
B+trees are commonly used in indexes for modern database management systems such as MYSQL, PostgreSQL, and SQLite. 

\subsection{Query Optimization}

Optimizing the SQL queries could also increase the performance of the database. Here are the 6 queries tune SQL queries. 
\url{https://www.geeksforgeeks.org/sql/sql-performance-tuning/}

\begin{itemize}
    \item SELECT fields instead of using SELECT *
    \item Avoid SELECT DISTINCT
    \item Use INNER JOIN instead of WHERE for Joins
    \item Use WHERE instead of HAVING
    \item Limit Wildcards to the end of a search term
    \item Use LIMIT for sampling query results
\end{itemize}

\subsection{N+1 problem}

The N+1 problem is a common problem when using the object relational mapping, or simply ORM. 
ORM is the process of abstracting the connection between programming language entities and their corresponding database elements \cite{Amazon_Web_Services}.
The definition of the problem is when an application retrieves a list, then performs additional queries for each item’s related data, which results in an inefficient query. 
(1 + N queries instead of optimized joins).  

Similar to the example above, assume there is a students table with keys student\_id, student\_name, and another table called courses with the keys course\_id, student\_id, and course\_name. 

If we were to get all students and courses they are taking in plain SQL, we would write something like this:

\begin{verbatim}
SELECT s.student_id, s.student_name, c.course_name
FROM students s
JOIN courses c ON s.student_id = c.student_id;
\end{verbatim}

Using an ORM, we could do something like this:

\begin{verbatim}
students = session.query(Student).all()
for s in students:
    for c in s.courses:
        print(s.name, c.name)
\end{verbatim}

This is where the N+1 problem happens. The way of getting all students and their courses through the ORM translates to the following raw SQL query: 

\begin{verbatim}
SELECT * FROM students;
SELECT * FROM courses WHERE student_id = 1;
SELECT * FROM courses WHERE student_id = 2;
SELECT * FROM courses WHERE student_id = 3;
\end{verbatim}

This is inefficient since we are performing 1 query (\texttt{SELECT * FROM students}),
and then performing $n$ queries (\texttt{SELECT * FROM course WHERE student\_id = n;}).
The performance of the database will get worse as the size of n increases.
It might be questioned why the ORM translates to inefficient SQL code, and the reason for this is lazy loading.
Lazy loading is when the application loads data only when the related data is accessed, and this is a default setting in many ORM frameworks.
In this example, the ORM does not know that all the students and their courses are going to be fetched.
They only understand each iteration of the nested for loop as a separate query and fetch the data for that specific iteration \cite{Jacky_2023}.
% https://dev.to/jackynote/solving-the-notorious-n1-problem-optimizing-database-queries-for-java-backend-developers-2o0p

There are a few ways to solve the N+1 problem, depending on the tools available by the ORM.
Most ORMs provide eager loading, which allows for fetching the related data right away instead of waiting until specific data is called.
This is the opposite of lazy loading. This reduces the need for additional queries and increases the efficiency when fetching data. Most ORMs
provide customization to decide when and how to perform eager loading, allowing optimization when querying databases.
For example, if we want to use eager loading in SQL Alchemy, we would do something like this:

\begin{verbatim}
from sqlalchemy.orm import joinedload

students = session.query(Student).options(joinedload(Student.courses)).all()
for s in students:
    for c in s.courses:
        print(s.name, c.name)
\end{verbatim}

The difference from the previous query is that instead of just fetching data from the students table,
this query fetches all course data related to the student beforehand to prevent lazy loading.

Another way to solve the N+1 problem is batch fetching. Batch fetching still keeps the idea of lazy loading,
but instead of fetching a single row of data, it fetches multiple rows of data at once. For example, 
we can write something like this in SQL Alchemy:

\begin{verbatim}
from sqlalchemy.orm import relationship

class Student(Base):
    __tablename__ = "students"
    id = Column(Integer, primary_key=True)
    name = Column(String)
    courses = relationship(
        "Course",
        back_populates="student",
        lazy="selectin",       # enable batch fetching
        batch_size=10          # load 10 parent IDs per batch
    )

students = session.query(Student).all()
\end{verbatim}

When creating the relation in SQLAlchemy, we can specify the batch size. 
In this example, by setting \texttt{lazy="selectin"} and \texttt{batch\_size=10}, 
we are telling the ORM when to fetch courses from students and to load 
10 student IDs per batch, resulting in the following query:

\begin{verbatim}
SELECT courses.id, courses.name, courses.student_id
FROM courses
WHERE courses.student_id IN (1,2,3,4,5,6,7,8,9,10);

-- Batch 2
SELECT courses.id, courses.name, courses.student_id
FROM courses
WHERE courses.student_id IN (11,12,13,14,15,16,17,18,19,20);
\end{verbatim}

This will avoid the N+1 problem by instead of fetching for n queries, it will fetch for m queries, where m is n/batch size.
Furthermore, batch fetching is useful when the size of the data becomes larger. Unlike eager loading, this avoids data duplication,
where we do not get the same student row repeated for every course.
Being able to adjust the batch size is also beneficial when scaling the application.
