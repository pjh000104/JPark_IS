\section[Frontend Design]{Frontend Design}\label{sec:newsec}
The frontend of the local cuisine application is responsible for accurately displaying the data sent from 
the backend. It is only responsible for presenting the data and managing user interactions,
while all the business logic and data persistence are handled by the backend. 
The tech stack used for the frontend development are Next.JS and Tailwind css. 

Next.js was selected as the primary frontend framework primarily for its strong support of modular, component-based development. 
As a React-based framework, it enables the application to be structured as a collection of independent, reusable components, 
each responsible for a specific portion of the user interface. This modular structure improves maintainability, 
as changes to one component can be made with minimal impact on others, and enhances scalability as new features 
can be added without restructuring the entire system.

The architectural philosophy of Next.js aligns closely with the modular backend implemented using Spring Boot. 
Just as the backend is organized into clearly separated modules and layers, 
the frontend is structured into distinct components and pages with well-defined responsibilities. 
For example, the frontend is divided into two primary modules: regions and cuisines, 
each responsible for handling its respective presentation logic and user interactions. 
This structure reflects the modular monolithic architecture of the backend, where related 
functionalities are grouped into modules while remaining part of a single deployable system. 
This design allows the similar domain across both layers, as the application maintains a consistent 
architectural model throughout the stack. 

Both technologies promote separation of concerns and low coupling between modules, 
resulting in a coherent and maintainable system design. 
This alignment between frontend and backend architecture enhances overall system clarity 
and simplifies future extension, testing, and maintenance.

\subsection{Frontend Architecture}

The frontend of the application is organized in a modular structure that mirrors the domain model of the backend. 
At the top level, the application is divided into two primary modules: regions and cuisines, 
each corresponding to a distinct functional area of the application. These separate directories in Next.js 
automatically define the routing structure, simplifying navigation and maintaining a clear correspondence 
between the file system and URL paths. For example, the folder app/cuisines/[cuisineName]/page.tsx 
automatically corresponds to the route /cuisines/{cuisineName}, where {cuisineName} 
is a dynamic segment representing the name of a specific cuisine. This structure allows the application 
to generate pages for each cuisine dynamically while preserving an intuitive and consistent URL hierarchy.

The regions module is responsible for presenting regional data and navigating between different regions, 
while the cuisines module handles the display of specific cuisines and their associated interactions. 
All logic related to reviews is included within the cuisines module, as reviews are inherently 
tied to individual cuisines. This ensures that each review is associated with its respective cuisine, 
maintaining a coherent modular structure. Consequently, the cuisines module encapsulates both 
the presentation of cuisine details and the user interactions related to reviews. This is because each review
as always associated with a certain cuisine.

Within each module, the application distinguishes between page components and reusable UI components. 
For example, the cuisines/[cuisineName] folder contains page-level components such as page.tsx, 
which serves as the base page for rendering the main cuisine information, along with supporting components 
like ReviewForm.tsx and ReviewList.tsx, which are attached to page.tsx to encapsulate specific UI functionalities. 
Similarly, the regions/[region] folder contains CuisineList.tsx, a reusable component for displaying 
a list of cuisines within a region. This separation allows components to be developed and maintained 
independently, enhancing code readability.

The application also employs a centralized API abstraction through the lib/api.ts file, 
which handles all communication with the backend REST endpoints. This approach ensures that 
data fetching logic is decoupled from presentation components, further reinforcing the separation 
of concerns. Global styling and layout are managed through globals.css and layout.tsx, 
providing consistent styling and structure across pages without introducing unnecessary complexity in 
individual components.


\subsection{API Communication}

The frontend communicates with the backend exclusively through RESTful APIs exposed by the Spring Boot application. 
All HTTP requests are centralized in the lib/api.ts file, which abstracts 
network communication and ensures that individual components remain focused on presentation and user interaction. 

This centralized API abstraction offers several advantages. First, it ensures that page components and 
reusable UI components do not directly manage network requests; they simply invoke the relevant function 
from api.ts and receive structured JSON data. Second, it improves maintainability: if a backend 
endpoint changes, only the corresponding function in api.ts needs to be updated. Third, 
it facilitates consistent error handling, as all functions check response statuses and log errors appropriately 

A concrete example of this abstraction can be seen in the function getCuisineByName(cuisineName) 
(Listing~\ref{lst:getCuisineByName}), which retrieves detailed information about a specific cuisine:

\begin{lstlisting}[caption={getCuisineByName -- fetching a single cuisine by name},
                   label={lst:getCuisineByName},
                   numbers=left,
                   firstnumber=1,
                   breaklines=true,
                   breakatwhitespace=true]
export async function getCuisineByName(cuisineName: string) {
  try {
    const res = await fetch(`${API_BASE_URL}/api/cuisines/${encodeURIComponent(cuisineName)}`, { credentials: "include" });
    if (!res.ok) { throw new Error(`Failed to fetch cuisine: ${cuisineName}`); }
    const data = await res.json();
    return data;
  } catch (error) {
    console.error("Error fetching cuisine:", error);
    return null;
  }
}
\end{lstlisting}

In this example, the function constructs the URL dynamically using the provided cuisineName parameter (line 3), 
performs a fetch request including credentials, checks for a successful response (line 4), and returns the JSON data (line 6). 
Any errors during the request are caught and logged (lines 7--9), ensuring consistent error handling throughout the frontend. 
The getCuisineByName function is called in the page.tsx in the cuisines module to load data related to a specific cuisine. 
The data returned includes the cuisine's unique identifier (id), the name of the cuisine (cuisineName), 
a textual description (description), the region it belongs to (regionName), and the country (country). 
This information is used by the cuisines module to render the main content of the page, displaying the 
cuisine's name, description, and region details to the user. Additionally, the cuisine id (id) is 
passed to the getReviewsByCuisineId function to fetch all reviews associated with this specific cuisine, 
ensuring that the displayed reviews are correctly linked to the selected cuisine and maintaining 
consistency between the main page content and the review components.

Similarly, creating a new review is abstracted through the createReview(data) function 
(Listing~\ref{lst:createReview}), which sends a POST request with a JSON payload:

\begin{lstlisting}[caption={createReview -- sending a new review to the backend},
                   label={lst:createReview},
                   numbers=left,
                   firstnumber=1,
                   breaklines=true,
                   breakatwhitespace=true]
export async function createReview(data: { cuisineId: number; rating: number; comment: string; }) {
  const response = await fetch(
    `${API_BASE_URL}/api/reviews`,
    {
      method: "POST",
      headers: { "Content-Type": "application/json" },
      body: JSON.stringify(data),
    }
  );
  if (!response.ok) { throw new Error("Failed to create review"); }
  return response.json();
}
\end{lstlisting}

This function takes a structured data object containing the cuisineId, rating, and comment (line 1), 
serializes it to JSON (line 5), and sends it to the backend (lines 2--6). It also checks for errors (line 7) 
and returns the response JSON (line 8). By using these abstractions, the frontend components remain 
clean and focused on rendering, while all network communication and error handling is handled in a 
single, maintainable location. 

In the application, the createReview function is invoked within the ReviewForm.tsx component 
of the cuisines module. When a user submits a new review through the form, the component collects 
the rating and comment values, along with the cuisineId from the currently displayed cuisine, 
and passes them to createReview. This ensures that the review is correctly associated with the selected 
cuisine and immediately reflected in the ReviewList component, maintaining a smooth user experience 
and a consistent connection between the cuisine data and its associated reviews.

This clear separation between data fetching and presentation reinforces the modular architecture 
of the frontend and maintains consistency with the backend's modular monolithic structure. 
By centralizing all network requests within api.ts, page and UI components focus solely on rendering 
and user interactions, while all communication with the backend and error handling is encapsulated 
in a single location. This approach enhances maintainability, ensures consistent data usage across 
components, and aligns the frontend and backend architectures, resulting in a coherent and extensible 
system design.

\subsection{Client-Side Caching}

As discussed in Section 4.4 (Make reference), client-side caching can significantly improve performance 
by reducing redundant network requests, and enable offline usability. 
However, since this application is primarily intended to run in an online environment 
with continuous communication between the frontend and backend, 
advanced frontend caching mechanisms were not fully implemented in the current version. 
Instead, the system relies on default browser caching behavior, 
while remaining open for future integration of additional client-side caching strategies.

The primary form of caching currently utilized in this project 
is the browser’s automatic caching of static frontend assets. 
When users access the application for the first time, 
the compiled Next.js JavaScript bundles (e.g., files under {/\_next/static/}) 
and the generated CSS stylesheet produced by Tailwind CSS 
are downloaded and stored in the browser cache.
Because these assets are static and shared across multiple pages, 
they do not need to be re-downloaded during subsequent navigation. 
For example, when a user navigates from the region listing page to a cuisine detail page, 
the previously loaded JavaScript bundle and stylesheet are reused from the browser cache. 
Only the dynamic data retrieved from the backend API needs to be fetched again. 
This reduces loading time and improves responsiveness without requiring additional manual configuration.

Although local storage was not directly implemented in the current version of the application, 
the existing frontend architecture allows for its straightforward integration. 
As described in Section~4.4, local storage enables persistent storage of lightweight, non-sensitive data on the client device 
across browser sessions.
Local storage could be used to retain user interface preferences in the future, such as theme selection, 
or to store recently viewed cuisines. 
It could also be applied to cache infrequently changing API responses 
in order to reduce repeated network requests. 

In addition to browser-level caching, HTTP Cache-Control headers could be configured on the backend 
to provide more control over resource freshness. 
For example, relatively stable data such as region or cuisine listings 
could be assigned appropriate caching directives (e.g., max-age) 
to allow temporary reuse without repeated server requests.
However, data that could change frequently such as user reviews would require stricter caching policies 
to ensure data consistency and accuracy. 
Although these cache-control strategies were not explicitly configured in the current implementation, 
the system architecture supports their future integration. 
This reflects a design consideration that balances performance optimization with data freshness, as discussed in Section~4.4.

\subsection{User Interface Design Choices}

