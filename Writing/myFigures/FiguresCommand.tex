\providecommand{\cacheAsideFigure}{
\begin{figure}[H]
    \centering
    \includegraphics[width=0.8\textwidth]{myFigures/cache-aside.png}
    \caption{Model of Cache-aside Pattern to explain common database caching strategies \cite{Thiraviyam_2025}.}
    \label{fig:cache-aside}
\end{figure}
}

\providecommand{\readThroughFigure}{
\begin{figure}[H]
    \centering
    \includegraphics[width=0.8\textwidth]{myFigures/read-through.png}
    \caption{Model of Read-Through Pattern to explain common database caching strategies \cite{Thiraviyam_2025}.}
    \label{fig:read-through}
\end{figure}
}

\providecommand{\writeAroundFigure}{
\begin{figure}[H]
    \centering
    \includegraphics[width=0.8\textwidth]{myFigures/write-around.png}
    \caption{Model of Write-Around Pattern to explain common database caching strategies \cite{Thiraviyam_2025}.}
    \label{fig:write-around}
\end{figure}
}

\providecommand{\writeBackFigure}{
\begin{figure}[H]
    \centering
    \includegraphics[width=0.8\textwidth]{myFigures/write-back.png}
    \caption{Model of Write-Back Pattern to explain common database caching strategies \cite{Thiraviyam_2025}.}
    \label{fig:write-back}
\end{figure}
}

\providecommand{\writeThroughFigure}{
\begin{figure}[H]
    \centering
    \includegraphics[width=0.8\textwidth]{myFigures/write-through.png}
    \caption{Model of Write-Through Pattern to explain common database caching strategies \cite{Thiraviyam_2025}.}
    \label{fig:write-through}
\end{figure}
}

\providecommand{\binaryTreeFigure}{
\begin{figure}[H]
    \centering
    \includegraphics[height=5cm]{myFigures/binaryTree.png}
    \caption{Model of a binary tree to explain datastructures used for database indexing.}
    \label{fig:binaryTree}
\end{figure}
}

\providecommand{\bTreeFigure}{
\begin{figure}[H]
    \centering
    \includegraphics[height=5cm]{myFigures/bTree.png}
    \caption{Model of a b tree to explain datastructures used for database indexing.}
    \label{fig:bTree}
\end{figure}
}

\providecommand{\bPlusTreeFigure}{
\begin{figure}[H]
    \centering
    \includegraphics[height=5cm]{myFigures/b+Tree.png}
    \caption{Model of a b+ tree to explain datastructures used for database indexing.}
    \label{fig:b+Tree}
\end{figure}
}

\providecommand{\leaderFollowerFigure}{
\begin{figure}[H]
    \centering
    \includegraphics[width=0.8\textwidth]{myFigures/leaderFollower.png}
    \caption{Figure of Leader and Follower replication strategy to explain the flow or write or read database querie \cite{Kleppmann_2017}.}
    \label{fig:leader follower}
\end{figure}
}

\providecommand{\erdFigure}{
\begin{figure}[p]  % 'p' = float on a page by itself
    \centering
    \includegraphics[width=\paperwidth,height=\paperheight,keepaspectratio]{myFigures/erd.png}
    \caption{Figure of the entity relationship diagram for the local cuisine application.}
    \label{fig:erd}
\end{figure}
}